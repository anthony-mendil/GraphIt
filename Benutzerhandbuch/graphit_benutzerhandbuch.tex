\documentclass[enabledeprecatedfontcommands,fontsize=11pt,paper=a4,twoside]{scrartcl}

\newcommand*{\red}{\textcolor{red}}
\newcommand*{\blue}{\textcolor{blue}}

\newcommand{\grad}{\ensuremath{^{\circ}} }
\renewcommand{\strut}{\vrule width 0pt height5mm depth2mm}

\usepackage[utf8]{inputenc}
\usepackage[T1]{fontenc}
\usepackage[final]{pdfpages}
% obere Seitenränder gestalten können
\usepackage{fancyhdr}
\usepackage{moreverb}
% Graphiken als jpg, png etc. einbinden können
\usepackage{graphicx}
\usepackage{stmaryrd}
% Floats Objekte mit [H] festsetzen
\usepackage{float}
% setzt URL's schön mit \url{http://bla.laber.com/~mypage}
\usepackage{url}
% Externe PDF's einbinden können
\usepackage{pdflscape}
% Verweise innerhalb des Dokuments schick mit " ... auf Seite ... "
% automatisch versehen. Dazu \vref{labelname} benutzen
\usepackage[ngerman]{varioref}
\usepackage[ngerman]{babel}
\usepackage{ngerman}
% Bibliographie
\usepackage{bibgerm}
\usepackage{svg}
% Tabellen
\usepackage{tabularx}
\usepackage{supertabular}
\usepackage[colorlinks=true, pdfstartview=FitV, linkcolor=blue,
            citecolor=blue, urlcolor=blue, hyperfigures=true,
            pdftex=true]{hyperref}
\usepackage{bookmark}



\newcounter{one}
\newcounter{two}[one]
\newcounter{three}[two]

\newcommand{\tone}{0\theone}
\newcommand{\ttwo}{0\thetwo}
\newcommand{\tthree}{0\thetree}
\newcommand{\one}{\stepcounter{one}0\theone}
\newcommand{\two}{\stepcounter{two}0\thetwo}
\newcommand{\three}{\stepcounter{three}0\thethree}
\newcommand\s{\rule{0pt}{4ex}}        
\newcommand{\cb}[1]{{\textcolor{blue}{#1}}}
\newcommand*{\hint}{\paragraph{\textcolor{blue}{Hinweise}}}
\newcommand*{\condition}{\paragraph{Voraussetzung}$\;$ \vspace{0.2cm}\\}
\newcommand*{\actions}{\paragraph{Vorgehen} $\;$\vspace{0.2cm}\\}

\usepackage{geometry}
\usepackage{hyperref}
\usepackage{pdfpages} 
\usepackage{colortbl}
%\usepackage{graphicx}   
%\usepackage[utf8x]{inputenc}
\usepackage{fmtcount}
\usepackage[ngerman]{babel}
\usepackage{booktabs}
\usepackage{fancyhdr}
\usepackage[T1]{fontenc}

\pagestyle{fancy}
\fancyhf{}

%

%

%

%

\definecolor{dartmouthgreen}{rgb}{0.05, 0.5, 0.06}
\definecolor{color}{rgb}{0.67, 0.88, 0.69}
\definecolor{todo}{rgb}{1.0, 0.41, 0.71}
\definecolor{sort}{rgb}{0.45, 0.76, 0.98}
\definecolor{prob}{rgb}{0.74, 0.83, 0.9}
\definecolor{anw}{rgb}{0.94, 0.86, 0.51}
\usepackage{amsmath}
\usepackage{tabularx}
\usepackage{setspace} 
\hypersetup{
	colorlinks = true,
	linkbordercolor = {white},
	linkcolor=dartmouthgreen,          % color of internal links (change box color with linkbordercolor)
	citecolor=red,        % color of links to bibliography
	filecolor=magenta,      % color of file links
	urlcolor=cyan 
}
\usepackage{geometry}
\geometry{
	a4paper,
	left=20mm,
	right=20mm,
	top=2cm,
	bottom=4cm,
	footskip=4cm
}


\addtolength{\headwidth}{20mm}
\addtolength{\headheight}{2\baselineskip}
\addtolength{\headheight}{0.61pt}


\renewcommand{\headrulewidth}{0pt}
\renewcommand{\headrule}{\vbox to 0pt{\rule{\headwidth}{0.2pt}}}
\setlength{\headsep}{30pt}


\hyphenation{Arbeits-paket}

% Damit Latex nicht zu lange Zeilen produziert:
%\sloppy
%Uneinheitlicher unterer Seitenrand:
%\raggedbottom

% Kein Erstzeileneinzug beim Absatzanfang
% Sieht aber nur gut aus, wenn man zwischen Absätzen viel Platz einbaut
%\setlength{\parindent}{0ex}

% Abstand zwischen zwei Absätzen
%\setlength{\parskip}{1ex}

% Seitenränder für Korrekturen verändern
%\addtolength{\evensidemargin}{-1cm}
%\addtolength{\oddsidemargin}{1cm}

%\bibliographystyle{gerapali}

% Lustige Header auf den Seiten
  \pagestyle{fancy}
  \setlength{\headheight}{70.55003pt}
  \fancyhead{}
  \fancyhead[LO,RE]{Software--Projekt 2\\ WiSe 2018/2019
  \\Architekturbeschreibung}
  \fancyhead[LE,RO]{Seite \thepage\\\slshape \leftmark\\\slshape \rightmark}

%
% Und jetzt geht das Dokument los....
%

\begin{document}

% Lustige Header nur auf dieser Seite
  \thispagestyle{fancy}
  \fancyhead[LO,RE]{ }
  \fancyhead[LE,RO]{Universität Bremen\\FB 3 -- Informatik\\
  Prof. Dr. Rainer Koschke \\TutorIn: Marcel Steinbeck}
  \fancyfoot[C]{}

% Start Titelseite
  \vspace{3cm}

  \begin{minipage}[H]{\textwidth}
  \begin{center}
  \bf
  \Large
  Software--Projekt 2 WiSe 2018/2019\\
  \smallskip
  \small
  VAK 03-BA-901.02\\
  \vspace{3cm}
  \end{center}
  \end{minipage}
  \begin{minipage}[H]{\textwidth}
  \begin{center}
  \vspace{1cm}
  \bf
  \Large Benutzerhandbuch\\
  \vspace{2cm}
  \includegraphics[width=3.0in]{logo_graphit.png}

  
  \end{center}
 
  \end{minipage}
  \vfill
  \begin{minipage}[H]{\textwidth}
  \begin{center}
  \sf
  \begin{tabular}{lr}
  Anthony Mendil & antmen@tzi.de \\
  Bastian Rexhäuser & brexhaeu@tzi.de\\
  Clement Phung & clement1@tzi.de \\
  Jacky Philipp Mach & machja@tzi.de \\
  Jonah Jaeger & jjaeger@tzi.de \\
  Nina Unterberg & nin\_unt@tzi.de \\
  \end{tabular}
  \\ ~
  \vspace{2cm}
  \\
  \it Abgabe: 10. März 2019 --- Version 1.0\\ ~
  \end{center}
  \end{minipage}

% Ende Titelseite

% Start Leerseite

\newpage

  \thispagestyle{fancy}
  \fancyhead{}
  \fancyhead[LO,RE]{Software--Projekt \\  2018/2019
  \\Benutzerhandbuch}
  
  \fancyhead[LE,RO]{Seite \thepage\\\slshape \leftmark\\~}
  \fancyfoot{}
  \renewcommand{\headrulewidth}{0.4pt}
  \tableofcontents

\newpage

  \fancyhead[LE,RO]{Seite \thepage\\\slshape \leftmark\\\slshape \rightmark}

%%%%%%%%%%%%%%%%%%%%%%%%%%%%%%%%%%%%%%%%%%%%%%%%%%%%%%%%%%%%%%%%%%%%%%%%
%%%%%%%%%%%%%%%%%%%%%%%%%%%%%%%%%%%%%%%%%%%%%%%%%%%%%%%%%%%%%%%%%%%%%%
\section*{Vorwort}
Mit diesem Handbuch wird die Bedienung der Software GraphIT beschrieben. Diese Anwendung wurde im Rahmen des Moduls \glqq{Softwareprojekt 2} \grqq im Wintersemester 2018/19 entwickelt wurde und ist zur Erstellung, Editierung und Visualisierung von Wirkungsdiagrammen für den Syndromansatz entworfen. Eine Nutzung des Programmes für Zwecke, die von dem angedachten Einsatzbereich abweicht ist möglich. Allerdings wird in diesem Dokument nicht weiter auf eine solche Nutzung eingegangen. Bei dieser Nutzung, die nicht dem angedachten Einsatz entspricht, ist ferner zu berücksichtigen, dass eventuell gewisse Funktionen wünschenswert wären, die aufgrund der Anforderungen aus der Anwenungsdomäne, für die diese Software entwickelt wurde, jedoch bewusst nicht eingebaut wurden. Insbesondere sind hier die Regeln zu beachten, die für die Erstellung von Wirkungszusammenhängen gemäß des Syndromansatzes gelten. Bei der vorliegenden Software handelt es sich um eine Neu-Entwicklung. Somit existieren keinerlei Vorversionen, mit denen der Funktionsumfang oder die Bedienung der Software in diesem Dokument verglichen werden könnte. 

%%%%%%%%%%%%%%%%%%%%%%%%%%%%%%%%%%%%%%%%%%%%%%%%%%%%%%%%%%%%%%%%%%%%%%%%
%%%%%%%%%%%%%%%%%%%%%%%%%%%%%%%%%%%%%%%%%%%%%%%%%%%%%%%%%%%%%%%%%%%%%%
\section{Einführung}
	\subsection{Adressierte Leser}
	Dieses Handbuch richtet sich an alle Nutzer, die die Software für ihren bestimmungsgemäßen Einsatz verwenden möchten. Dieser umfasst die Erstellung, Bearbeitung und Auswertung von Wirkungszusammenhängen gemäß des Syndromansatzes. Es ist sowohl für Anfänger geeignet, die noch nie oder nur sehr selten mit einem vergleichbaren Programm gearbeitet haben, als auch für Fortgeschrittene und Experten, die über mehr Erfahrung in der Bedienung ähnlicher Software verfügen. Es existieren keine weiteren Benutzerhandbücher neben diesem Dokument. Dementsprechend finden sich in diesem Dokument auch für alle Nutzer der thematisierten Software Hinweise zur korrekten Verwendung der Software. Von dem Adressatenkreis des vorliegenden Dokuments wird die Kenntnis der Terminologie des Syndromansatzes erwartet.  
	\subsection{Verwandte Dokumente}
	\subsection{Konventionen}
	\subsection{Informationen über die Verwendung des Dokuments}

\begin{itemize}
	\item screenshot auf windows	
\end{itemize}


%%%%%%%%%%%%%%%%%%%%%%%%%%%%%%%%%%%%%%%%%%%%%%%%%%%%%%%%%%%%%%%%%%%%%%
\newpage
\section{Installation und Starten des Programms} \label{sec:installation}

	\subsection{Installationsvoraussetzungen}

	Zur Nutzung müssen auf dem Nutzercomputer zwei \blue{Programme} installiert sein.

	\subsubsection{Java 8}
	\subsubsection{Maven}
	Das weiß Jonah

	\subsection{Installationsanweisungen}
	\subsection{Starten des Programms}
\section{Benutzeroberfläche}
	\subsection{Tasten-/ Mausbelegungen}
	\subsection{Übersicht}
	
	
%%%%%%%%%%%%%%%%%%%%%%%%%%%%%%%%%%%%%%%%%%%%%%%%%%%%%%%%%%%%%%%%%%%%%%	
\section{Übersicht} \label{sec:uebersicht}
\begin{itemize}
	\item Aufzählung und grundlegende Erklärung der Funktionsmodi
\end{itemize}


\section{Instruktionen zur Nutzung des Programms} \label{sec:nutzung}
	\subsection{Auswahl von Graphelementen}
		\subsubsection{Eines Elements}
		\subsubsection{Mehrerer Elemente}
	\subsection{Editierung des Graphen}
		\paragraph{\blue{Hinweis}}
		\begin{itemize}
			\item In den folgenden Anleitungen wird immer davon ausgegangen, dass sich der Benutzer im Bearbeiter-/ Erstellermodus befindet und er somit die Berechtigung hat den Graphen zu editieren.
		\end{itemize}
		\subsubsection{Undo}
		\subsubsection{Redo}
		\subsubsection{Sphäre hinzufügen}	
		\condition 	
		Das Programm ist gestartet.
		\actions
		\begin{enumerate}
			\item In der Menüleiste den Button \textit{Sphäre hinzufügen} durch einen Links-Klick aktivieren.
			\item Den Cursor auf die gewünschte Position bewegen, an der sich keine andere Sphäre befindet, und auf die linke Maustaste klicken.
		\end{enumerate}
		\hint
		\begin{itemize}
			\item Es ist nicht möglich eine Sphäre auf einer anderen Sphäre hinzuzufügen. Wird dies versucht, wird keine neue Sphäre hinzugefügt und es erscheint eine Fehlermeldung.
	\item Es ist nicht möglich eine Sphäre hinzuzufügen, wenn die Anzahl an Sphären im Graphen bereits der maximalen Anzahl an Sphären, die im Erstellermodus in den Vorlage-Regeln festgelegt wurde, entspricht und man im Bearbeitermodus ist. Wird dies versucht, wird kein neues Symtom hinzugefügt und es erscheint eine Fehlermeldung.
	\end{itemize}
		\subsubsection{Sphäre entfernen}
			\condition 	
		Ein Syndromansatz mit mindestens einer Sphäre ist im Programm geöffnet. 
		\actions  
		Alternative 1:
		\begin{enumerate}
			\item Die Sphäre, die gelöscht werden soll mit einem Links-Klick auswählen.
			\item Die /textit{Entfernen}-Teste der Tastatur drücken.
		\end{enumerate}
		Alternative 2:
			\begin{enumerate}
			\item Die Sphäre, die gelöscht werden soll mit einem Links-Klick auswählen.
			\item In der Menüleiste einen Links-Klick auf den Button \textit{Sphäre löschen} ausführen.
		\end{enumerate}
		Alernative 3:
		\begin{enumerate}
			\item Mit einem Rechts-Klick auf die Sphäre, die gelöscht werden soll, das Kontextmenü öffnen. 
			\item In diesem Kontextmenü ganz unten die \textit{Entfernen}-Option mit einem Links-Klick auswählen.
		\end{enumerate}
		\hint
		\begin{itemize}
			\item Wird eine Sphäre gelöscht, so werden automatisch auch alle Symptome entfernt, die zu dieser Sphäre gehören. Damit werden dann auch alle Relationen gelöscht, die in einem dieser Symptome einmünden oder von einem dieser Symptome ausgehen.
			\item Es ist nicht möglich eine Sphäre zu löschen, wenn der Titel und/oder die Position und/oder der Style der zu löschende Sphäre im Erstellermodus in den Vorlage-Regeln gelocked wurde. Wird dies versucht, wird die Sphäre nicht gelöscht und es erscheint eine Fehlermeldung.
		\end{itemize}
		\subsubsection{Sphäre verschieben}
				\condition 	
		Ein Syndrom mit mindestens einer Sphäre ist im Programm geöffnet. 
		\actions  
		\begin{enumerate}
			\item Die Sphäre, die verschoben werden soll mit einem Rechts-Klick auswählen.
			\item Die (rechte) Taste gedrückt halten und den Cursor an eine Zielstelle bewegen, an der sich keine andere Sphäre befindet. Beim Bewegen des Cursors bewegt sich die Sphäre bereits mit. 
			\item Die rechte Maustaste loslassen.
		\end{enumerate}
		\hint
		\begin{itemize}
			\item Es ist nicht möglich eine Sphäre zu verschieben, wenn sich an der Zielposition bereits eine andere Sphäre befindet. Wird dies versucht, wird die Sphäre nicht gelöscht und es erscheint eine Fehlermeldung.
			\item Es ist nicht möglich eine Sphäre im Bearbeitermodus zu verschieben, wenn die Position dieser Sphäre im Erstellermodus in den Vorlage-Regeln gelocked wurde. Wird dies versucht, wird die Sphäre nicht gelöscht und es erscheint eine Fehlermeldung.
		\end{itemize}
		\subsubsection{Größe der Sphäre verändern}
				\condition 	
		Ein Syndrom mit mindestens einer Sphäre ist im Programm geöffnet. 
		\actions  
		Alternative 1:
		\begin{enumerate}
			\item Die Sphäre, deren Größe verändert werden soll werden soll, mit einem Links-Klick auswählen.
			\item So oft die \glqq\textit{+}\grqq- / \glqq\textit{-}\grqq-Taste der Tastatur drücken bis die Sphäre die gewünschte Größe hat. \\(Nicht die \glqq\textit{+}\grqq- / \glqq\textit{-}\grqq-Taste des Nummernblocks)
		\end{enumerate}
		Alternative 2:
			\begin{enumerate}
			\item Die Sphäre, die gelöscht werden soll mit einem Links-Klick auswählen.
			\item In der Menüleiste so oft  Links-Klick auf den Button \textit{Sphäre vergrößern}/\textit{Sphäre verkleinern} ausführen bis die Sphäre die gewünschte Größe hat.. 
		\end{enumerate}
		\hint
		\begin{itemize}
			\item Eine Sphäre kann nicht mehr verkleinert werden, wenn sie bereits ihre minimale Größe hat.
			\item Eine Sphäre kann nicht weiter vergrößert werden, wenn die Vergrößerung zu einer Überlappung von zwei Sphären führen würde.
			\end{itemize}
		\subsubsection{Farbe der Sphäre ändern}
				\condition 	
		Ein Syndrom mit mindestens einer Sphäre ist im Programm geöffnet. 
		\actions  
		Alternative 1:
		\begin{enumerate}
			\item In der Menüleiste mit der linken Maustaste auf den Button \textit{Hintergrundfarbe der Sphäre verändern} klicken.
			\item Mit einem Rechts-Klick auf die Sphäre, deren Farbe geändert werden soll, ihr Kontextmenü öffnen.
			\item im Kontextmenü den Punkt \textit{Farbe} auswählen.
		\end{enumerate}
		Alternative 2:
		\begin{enumerate}
			\item In der Menüleiste mit der linken Maustaste auf den Button \textit{Hintergrundfarbe der Sphäre verändern} klicken.
			\item In dem sich öffnenden Fenster \textit{Costom Color} mit einem Lnks-Klick auswählen und in dem sich öffnenden Farbwahl-Fenster eine Farbe einstellen und diese mit der \textit{Enter}-Taste bestätigen.
			 \item Mit einem Rechts-Klick auf die Sphäre, deren Farbe geändert werden soll, ihr Kontextmenü öffnen.
			\item im Kontextmenü den Punkt \textit{Farbe} auswählen.
		\end{enumerate}
		Alternative 3:
		\begin{enumerate}
			\item Die Sphäre, deren Farbe geändert werden soll, mit einem Links-Klick auswählen.
			\item In der Menüleiste mit der linken Maustaste auf den Button \textit{Hintergrundfarbe der Sphäre verändern} klicken.
			\item In dem sich öffnenden Fenster die gewünschte Farbe mit einem Lnks-Klick auswählen.
		\end{enumerate}
		Alternative 4:
			\begin{enumerate}
			\item Die Sphäre, deren Farbe geändert werden soll, mit einem Links-Klick auswählen.
			\item In der Menüleiste mit der linken Maustaste auf den Button \textit{Hintergrundfarbe der Sphäre verändern} klicken.
			\item In dem sich öffnenden Fenster \textit{Costom Color} mit einem Lnks-Klick auswählen und in dem sich öffnenden Farbwahl-Fenster eine Farbe einstellen und diese mit der \textit{Enter}-Taste bestätigen.
		\end{enumerate}
		\hint
		\begin{itemize}
			\item Die Bedienung des Farbwahl-Fensters, zu dem man über \textit{Custom Color} gelangt, ist im Kapiel -----XXX------ beschrieben.
			\item Es ist nicht möglich die Farbe einer Sphäre zu ändern, wenn der Style dieser Sphäre im Erstellermodus in den Vorlage-Regeln gelocked wurde. Wird dies versucht, wird die Sphäre nicht gelöscht und es erscheint eine Fehlermeldung.
	\end{itemize}	
	\subsubsection{Titel der Sphäre ändern}
				\condition 	
		Ein Syndrom mit mindestens einer Sphäre ist im Programm geöffnet. 
		\actions  
		\begin{enumerate}
			\item Die Sphäre, deren Titel geändert werden soll mit einem Rechts-Klick anklicken, sodass sich ihr Kontext-Menü öffnet.
			\item Im Kontextmenü ganz oben \textit{Titel} mit einem Links-Klick auswählen. 
			\item In dem sich öffnenden Fenster für die gewünschten Sprache(n) den nuen Titel in die dafür vorgesehenen Felder eingeben.
			\item Die Änderung des Titels mit einem Links-Klick auf die \textit{Speichern}-Schaltfläche abschließen.
		\end{enumerate}
		\hint
		\begin{itemize}
			\item Der Titel einer Sphäre darf kein Semikolon enthalten.
			\item 
Es ist nicht möglich den Titel einer Sphäre zu ändern, wenn der Titel dieser Sphäre im Erstellermodus in den Vorlage-Regeln gelocked wurde. Wird dies versucht, wird die Sphäre nicht gelöscht und es erscheint eine Fehlermeldung.
		\end{itemize}
		\subsubsection{Contextmenü einer Sphäre öffnen/benutzen}
				\condition 	
		Ein Syndrom mit mindestens einer Sphäre ist im Programm geöffnet. 
		\actions  
		Alternative 1:
		\begin{enumerate}
			\item Auf einer Sphäre einen Rechts-Klick ausführen.
			\item im sich öffnenden Kontextmenü die gewünschte Option mit einem Links-Klick auswählen.
		\end{enumerate}
		\hint
		\begin{itemize}
			\item Es ist egal, ob eine die Späre, deren Kontextmenü geöffnet werden soll, vor dem Rechts-Klick auf diese Sphäre ausgewählt worden ist oder nicht.
		\end{itemize}
		\subsubsection{Schriftart/-größe der Beschriftung der Sphäre ändern}
				\condition 	
		Ein Syndrom mit mindestens einer Sphäre ist im Programm geöffnet. 
		\actions  
		Alternative 1:
		\begin{enumerate}
			\item Die Sphäre, für die die Schriftart/-größe geändert werden soll, mit einem Links-Klick auswählen.
			\item In der Menüleiste einen Links-Klick auf das Feld klicken, in dem de aktuelle Schriftart/-größe der Sphäre angezeigt wird.
			\item In dem Drop-Down-Menü die Schrifart/-größe auswählen, die die Beschriftung der Sphäre haben soll.
		\end{enumerate}
		Alternative 2:
			\begin{enumerate}
			\item In der Menüleiste einen Links-Klick auf das Feld klicken, in dem de aktuelle Schriftart/-größe der Sphäre angezeigt wird.
			\item In dem Drop-Down-Menü die Schrifart/-größe auswählen, die die Beschriftung der Sphäre haben soll.
			\item Mit einem Rechts-Klick auf die Sphäre klicken, deren Schriftart/-größe geändert werden soll.
			\item In dem sich öffnenden Kontextmenü die Option \textit{Schriftart}/\textit{Schriftgröße} auswählen.
		\end{enumerate}
		\hint
		\begin{itemize}
			\item Die Schriftart/-größe kann nicht geändert
			\item Es ist nicht möglich die Schriftart/-größe einer Sphäre zu ändern, wenn der Style dieser Sphäre im Erstellermodus in den Vorlage-Regeln gelocked wurde. Wird dies versucht, wird die Sphäre nicht gelöscht und es erscheint eine Fehlermeldung.
			\end{itemize}
		\subsubsection{Sphären layouten}
				\condition 	
		Ein Syndrom mit mindestens einer Sphäre ist im Programm geöffnet. 
		\actions  
		\begin{enumerate}
			\item In der Menüleiste im Bereich Sphären auf den \textit{Automatische Anordnung}.Button mit Links-Klick klicken.
		\end{enumerate}
		\hint
		\begin{itemize}
			\item Es ist nicht möglich eine Sphäre zu löschen, wenn der Titel und/oder die Position und/oder der Style der zu löschende Sphäre im Erstellermodus in den Vorlage-Regeln gelocked wurde. Wird dies versucht, wird die Sphäre nicht gelöscht und es erscheint eine Fehlermeldung.
			\end{itemize}
		\newpage
		\subsubsection{Symptom hinzufügen}
		\condition 	
		Ein Syndrom mit mindestens einer Sphäre ist im Programm geöffnet. 
		\actions
		\begin{enumerate}
			\item In der Menüleiste den Button \textit{Symptom hinzufügen} durch einen Links-Klick aktivieren.
			\item Den Cursor auf die gewünschte Position innerhalb einer Sphäre bewegen, an der sich kein(e) andere(s) Symptom / Relation befindet, und auf die linke Maustaste klicken.
		\end{enumerate}
		\hint
		\begin{itemize}
			\item Es ist nicht möglich ein Symptom auf einen anderem Symptom hinzuzufügen. Wird dies versucht, wird kein neues Symtom hinzugefügt und es erscheint eine Fehlermeldung.
			\item Es ist nicht möglich ein Symptom außerhalb einer Sphäre hinzuzufügen. Wird dies versucht, wird kein neues Symtom hinzugefügt und es erscheint eine Fehlermeldung.
			\item Es ist nicht möglich ein Symptom auf einer Relation hinzuzufügen. Wird dies versucht, wird kein neues Symtom hinzugefügt und es erscheint eine Fehlermeldung.
			\item Es ist nicht möglich ein Symptom hinzuzufügen, wenn die Anzahl an Symptomen im Graphen bereits der maximalen Anzahl an Symptomen, die im Erstellermodus in den Vorlage-Regeln festgelegt wurde, entspricht und man im Bearbeitermodus ist. Wird dies versucht, wird kein neues Symtom hinzugefügt und es erscheint eine Fehlermeldung.
			\item Es ist nicht möglich ein Symptom hinzuzufügen, wenn die Anzahl an Symptomen in der Sphäre, in der man ein Symptom hinzufügen möchte, bereits der maximalen Anzahl an Symptomen für diese Sphäre, die im Erstellermodus in den Vorlage-Regeln festgelegt wurde, entspricht und man im Barbeiterodus ist. Wird dies versucht, wird kein neues Symtom hinzugefügt und es erscheint eine Fehlermeldung.		
		\end{itemize}
		\newpage
		\subsubsection{Symptom entfernen}
		\subsubsection{Symptom verschieben}
		\subsubsection{Größe eines Symptoms verändern}
		\subsubsection{Füllfarbe eines Symptoms verändern}
		\subsubsection{Randfarbe eines Symptoms verändern}
		\subsubsection{Titel eines Symptoms ändern}
		\subsubsection{Form eines Symptoms verändern}
		\subsubsection{Schriftart-/ größe der Beschriftung eines Symptoms ändern}
		\subsubsection{Symptome layouten}
			
		\subsubsection{Relation hinzufügen}
		\subsubsection{Relation entfernen}
		\subsubsection{Ankerpunkte einblenden}
		\subsubsection{Ankerpunkte hinzufügen}
		\subsubsection{Ankerpunkte entfernen}
		\subsubsection{Farbe einer Relation verändern}
		\subsubsection{Type einer Relation verändern}
		
		\subsubsection{Graphelemente hervorheben}
		\subsubsection{Graphelemente Hervorhebung hinzufügen}
		\subsubsection{Graphelemente Hervorhebung entfernen}
		
		\subsubsection{Graphelemente ausblenden/ einblenden}
		\subsubsection{Graphelemente Fadeout hinzufügen}
		\subsubsection{Graphelemente Fadeout entfernen}
		
		\subsubsection{Contextmenü}
	
	\subsection{Übersichtsleiste}
		\subsubsection{Filtern}
		\subsubsection{Elemente auswählen}
		\subsubsection{Contextmenü öffnen}
		
	\subsection{Zoom}
		\subsubsection{Zoom}
		\subsubsection{Zoom.Kontext}
	\subsection{Allgemeine Einstellungen}
		\subsubsection{Sprache der Benutzeroberfläche}
		\subsubsection{Sprache der Beschriftung der Graphelement}
	\subsection{Import/ Öffnen}
		\subsubsection{Neue Datei}
		\subsubsection{Datei öffnen}
		\subsubsection{GXL}
		\subsubsection{oof}
	\subsection{Export/ Speichern}
		\subsubsection{Speichern unter}
		\subsubsection{GXL}
		\subsubsection{oof}
		\subsubsection{Pdf}
		\subsubsection{Verlaufsprotokoll}
	\subsection{Drucken}
	\subsection{Analysefunktionen}
		\subsubsection{Graphmaße}
		\subsubsection{Vorgänger/ Nachfolger von Symptom(en) hervorheben}
		\subsubsection{Kürzester Pfad zw. Symptomen}
		\subsubsection{Alle Pfade zw. Symptomen}
		\subsubsection{Pfeilketten}
		\subsubsection{Konvergente Verzweigungen}
		\subsubsection{Divergente Verzweigungen}
		\subsubsection{Zyklen}
		\subsubsection{Alle Relationen eines Typs hervorheben}		
	\subsection{Verlauf}
		\subsubsection{Anzeigen}
		\subsubsection{Filtern}
	\subsection{Vorlage}
		\subsubsection{Element spezifische Vorlagereglen}
		\subsubsection{Graph spezifische Vorlagereglen}
		\subsubsection{Vorlage erstellen}
		\subsubsection{Vorlage verwenden}
		\subsubsection{Vorlage exportieren}
	\subsection{Dialogfenster}
	\newpage
	\section{Fehlermeldungen} \label{sec:fehlermeldungen}
	
	%die verschiedenen Fehlermeldungen, die der Nutzer bekommen kann erklären als subsec
	
%%%%%%%%%%%%%%%%%%%%%%%%%%%%%%%%%%%%%%%%%%%%%%%%%%%%%%%%%%%%%%%%%%%%%%
\section{Warnhinweise} \label{sec:warnhinweise}
	
	
	
	
%%%%%%%%%%%%%%%%%%%%%%%%%%%%%%%%%%%%%%%%%%%%%%%%%%%%%%%%%%%%%%%%%%%%%%
\section{FAQ}
\newpage
%%%%%%%%%%%%%%%%%%%%%%%%%%%%%%%%%%%%%%%%%%%%%%%%%%%%%%%%%%%%%%%%%%%%%%
\section{Anhang} \label{sec:anhang}	
	\subsection{Glossar}
	
	\begin{description}
		\item[Syndromansatz] Graph bla bla
	\end{description}
	
	\subsection{Abbildungsverzeichnis}
	\listoffigures
	

\newpage



%%%%%%%%%%%%%%%%%%%%%%%%%%%%%%%%%%%%%%%%%%%%%%%%%%%%%%%%%%%%%%%%%%%%%%

\end{document}


