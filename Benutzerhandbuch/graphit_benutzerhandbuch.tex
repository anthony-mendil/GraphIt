\documentclass[enabledeprecatedfontcommands,fontsize=11pt,paper=a4,twoside]{scrartcl}

\newcommand*{\red}{\textcolor{red}}
\newcommand*{\blue}{\textcolor{blue}}

\newcommand{\grad}{\ensuremath{^{\circ}} }
\renewcommand{\strut}{\vrule width 0pt height5mm depth2mm}

\usepackage[utf8]{inputenc}
\usepackage[T1]{fontenc}
\usepackage[final]{pdfpages}
% obere Seitenränder gestalten können
\usepackage{fancyhdr}
\usepackage{moreverb}
% Graphiken als jpg, png etc. einbinden können
\usepackage{graphicx}
\usepackage{stmaryrd}
% Floats Objekte mit [H] festsetzen
\usepackage{float}
% setzt URL's schön mit \url{http://bla.laber.com/~mypage}
\usepackage{url}
% Externe PDF's einbinden können
\usepackage{pdflscape}
% Verweise innerhalb des Dokuments schick mit " ... auf Seite ... "
% automatisch versehen. Dazu \vref{labelname} benutzen
\usepackage[ngerman]{varioref}
\usepackage[ngerman]{babel}
\usepackage{ngerman}
% Bibliographie
\usepackage{bibgerm}
\usepackage{svg}
% Tabellen
\usepackage{tabularx}
\usepackage{supertabular}
\usepackage[colorlinks=true, pdfstartview=FitV, linkcolor=blue,
            citecolor=blue, urlcolor=blue, hyperfigures=true,
            pdftex=true]{hyperref}
\usepackage{bookmark}



\newcounter{one}
\newcounter{two}[one]
\newcounter{three}[two]

\newcommand{\tone}{0\theone}
\newcommand{\ttwo}{0\thetwo}
\newcommand{\tthree}{0\thetree}
\newcommand{\one}{\stepcounter{one}0\theone}
\newcommand{\two}{\stepcounter{two}0\thetwo}
\newcommand{\three}{\stepcounter{three}0\thethree}
\newcommand\s{\rule{0pt}{4ex}}        
\newcommand{\cb}[1]{{\textcolor{blue}{#1}}}
\newcommand*{\hint}{\paragraph{\textcolor{blue}{Hinweise}}}
\newcommand*{\condition}{\paragraph{Voraussetzung}$\;$ \vspace{0.2cm}\\}
\newcommand*{\actions}{\paragraph{Vorgehen}}

\usepackage{geometry}
\usepackage{hyperref}
\usepackage{pdfpages} 
\usepackage{colortbl}
%\usepackage{graphicx}   
%\usepackage[utf8x]{inputenc}
\usepackage{fmtcount}
\usepackage[ngerman]{babel}
\usepackage{booktabs}
\usepackage{fancyhdr}
\usepackage[T1]{fontenc}

\pagestyle{fancy}
\fancyhf{}

%

%

%

%

\definecolor{dartmouthgreen}{rgb}{0.05, 0.5, 0.06}
\definecolor{color}{rgb}{0.67, 0.88, 0.69}
\definecolor{todo}{rgb}{1.0, 0.41, 0.71}
\definecolor{sort}{rgb}{0.45, 0.76, 0.98}
\definecolor{prob}{rgb}{0.74, 0.83, 0.9}
\definecolor{anw}{rgb}{0.94, 0.86, 0.51}
\usepackage{amsmath}
\usepackage{tabularx}
\usepackage{setspace} 
\hypersetup{
	colorlinks = true,
	linkbordercolor = {white},
	linkcolor=dartmouthgreen,          % color of internal links (change box color with linkbordercolor)
	citecolor=red,        % color of links to bibliography
	filecolor=magenta,      % color of file links
	urlcolor=cyan 
}
\usepackage{geometry}
\geometry{
	a4paper,
	left=20mm,
	right=20mm,
	top=2cm,
	bottom=4cm,
	footskip=4cm
}


\addtolength{\headwidth}{20mm}
\addtolength{\headheight}{2\baselineskip}
\addtolength{\headheight}{0.61pt}


\renewcommand{\headrulewidth}{0pt}
\renewcommand{\headrule}{\vbox to 0pt{\rule{\headwidth}{0.2pt}}}
\setlength{\headsep}{30pt}


\hyphenation{Arbeits-paket}

% Damit Latex nicht zu lange Zeilen produziert:
%\sloppy
%Uneinheitlicher unterer Seitenrand:
%\raggedbottom

% Kein Erstzeileneinzug beim Absatzanfang
% Sieht aber nur gut aus, wenn man zwischen Absätzen viel Platz einbaut
%\setlength{\parindent}{0ex}

% Abstand zwischen zwei Absätzen
%\setlength{\parskip}{1ex}

% Seitenränder für Korrekturen verändern
%\addtolength{\evensidemargin}{-1cm}
%\addtolength{\oddsidemargin}{1cm}

%\bibliographystyle{gerapali}

% Lustige Header auf den Seiten
  \pagestyle{fancy}
  \setlength{\headheight}{70.55003pt}
  \fancyhead{}
  \fancyhead[LO,RE]{Software--Projekt 2\\ WiSe 2018/2019
  \\Architekturbeschreibung}
  \fancyhead[LE,RO]{Seite \thepage\\\slshape \leftmark\\\slshape \rightmark}

%
% Und jetzt geht das Dokument los....
%

\begin{document}

% Lustige Header nur auf dieser Seite
  \thispagestyle{fancy}
  \fancyhead[LO,RE]{ }
  \fancyhead[LE,RO]{Universität Bremen\\FB 3 -- Informatik\\
  Prof. Dr. Rainer Koschke \\TutorIn: Marcel Steinbeck}
  \fancyfoot[C]{}

% Start Titelseite
  \vspace{3cm}

  \begin{minipage}[H]{\textwidth}
  \begin{center}
  \bf
  \Large
  Software--Projekt 2 WiSe 2018/2019\\
  \smallskip
  \small
  VAK 03-BA-901.02\\
  \vspace{3cm}
  \end{center}
  \end{minipage}
  \begin{minipage}[H]{\textwidth}
  \begin{center}
  \vspace{1cm}
  \bf
  \Large Benutzerhandbuch\\
  \vspace{2cm}
  \includegraphics[width=3.0in]{logo_graphit.png}

  
  \end{center}
 
  \end{minipage}
  \vfill
  \begin{minipage}[H]{\textwidth}
  \begin{center}
  \sf
  \begin{tabular}{lr}
  Anthony Mendil & antmen@tzi.de \\
  Bastian Rexhäuser & brexhaeu@tzi.de\\
  Clement Phung & clement1@tzi.de \\
  Jacky Philipp Mach & machja@tzi.de \\
  Jonah Jaeger & jjaeger@tzi.de \\
  Nina Unterberg & nin\_unt@tzi.de \\
  \end{tabular}
  \\ ~
  \vspace{2cm}
  \\
  \it Abgabe: 10. März 2019 --- Version 1.0\\ ~
  \end{center}
  \end{minipage}

% Ende Titelseite

% Start Leerseite

\newpage

  \thispagestyle{fancy}
  \fancyhead{}
  \fancyhead[LO,RE]{Software--Projekt \\  2018/2019
  \\Benutzerhandbuch}
  
  \fancyhead[LE,RO]{Seite \thepage\\\slshape \leftmark\\~}
  \fancyfoot{}
  \renewcommand{\headrulewidth}{0.4pt}
  \tableofcontents

\newpage

  \fancyhead[LE,RO]{Seite \thepage\\\slshape \leftmark\\\slshape \rightmark}


%%%%%%%%%%%%%%%%%%%%%%%%%%%%%%%%%%%%%%%%%%%%%%%%%%%%%%%%%%%%%%%%%%%%%%%%
%%%%%%%%%%%%%%%%%%%%%%%%%%%%%%%%%%%%%%%%%%%%%%%%%%%%%%%%%%%%%%%%%%%%%%
\section{Einführung}
	\subsection{Adressierte Leser}
	\subsection{Verwandte Dokumente}
	\subsection{Konventionen}
	\subsection{Informationen über die Verwendung des Dokuments}

\begin{itemize}
	\item screenshot auf windows	
\end{itemize}


%%%%%%%%%%%%%%%%%%%%%%%%%%%%%%%%%%%%%%%%%%%%%%%%%%%%%%%%%%%%%%%%%%%%%%
\newpage
\section{Installation und Starten des Programms} \label{sec:installation}

	\subsection{Installationsvoraussetzungen}

	Zur Nutzung müssen auf dem Nutzercomputer zwei \blue{Programme} installiert sein.

	\subsubsection{Java 11}
	\subsubsection{Maven}
	Das weiß Jonah

	\subsection{Installationsanweisungen}
	\subsection{Starten des Programms}
\section{Benutzeroberfläche}
	\subsection{Tasten-/ Mausbelegungen}
	\subsection{Übersicht}
	
	
%%%%%%%%%%%%%%%%%%%%%%%%%%%%%%%%%%%%%%%%%%%%%%%%%%%%%%%%%%%%%%%%%%%%%%	
\section{Übersicht} \label{sec:uebersicht}
\begin{itemize}
	\item aufzählung der modi und grundlegende erklärung
\end{itemize}


\section{Instruktionen zur Nutzung des Programms} \label{sec:nutzung}
	\subsection{Auswahl von Graphelementen}
		\subsubsection{Eines Elements}
		\subsubsection{Mehrerer Elemente}
	\subsection{Editierung des Graphen}
		\paragraph{\blue{Hinweis}}
		\begin{itemize}
			\item In den folgenden Anleitungen wird immer davon ausgegangen, dass sich der Benutzer im Bearbeiter-/ Erstellermodus befindet und er somit die Berechtigung hat den Graphen zu editieren.
		\end{itemize}
		\subsubsection{Undo}
		\subsubsection{Redo}
		\subsubsection{Sphäre hinzufügen}		
		\subsubsection{Sphäre entfernen}
		\subsubsection{Sphäre verschieben}
		\subsubsection{Größe der Sphäre verändern}
		\subsubsection{Farbe ändern der Sphäre}
		\subsubsection{Titel der Sphäre ändern}
		\subsubsection{Contextmenü einer Sphäre öffnen/ benutzen}
		\subsubsection{Schriftart-/ größe der Beschriftung der Sphäre ändern}
		\subsubsection{Sphären layouten}
		\newpage
		\subsubsection{Symptom hinzufügen}
		\condition 	
		Ein Syndromansatz mit mindestens einer Sphäre ist im Programm geöffnet. 
		\actions
		\begin{enumerate}
			\item In der Menüleiste den Button \textit{Symptom hinzufügen} durch einen Klick aktivieren.
			\item Auf die gewünschte Position innerhalb einer Sphäre drücken, wo sich kein(e) anderes Symptom/ Relation befindet, klicken.
		\end{enumerate}
		\hint
		\begin{itemize}
			\item Es ist nicht möglich ein Symptom auf einen anderem Symptom hinzuzufügen. Wird dies versucht, wird kein neues Symtom hinzugefügt und es erscheint eine Fehlermeldung.
			\item Es ist nicht möglich ein Symptom außerhalb einer Sphäre hinzuzufügen. Wird dies versucht, wird kein neues Symtom hinzugefügt und es erscheint eine Fehlermeldung.
			\item Es ist nicht möglich ein Symptom auf einer Relation hinzuzufügen. Wird dies versucht, wird kein neues Symtom hinzugefügt und es erscheint eine Fehlermeldung.
		
		\end{itemize}
		\newpage
		\subsubsection{Symptom entfernen}
		\subsubsection{Symptom verschieben}
		\subsubsection{Größe eines Symptoms verändern}
		\subsubsection{Füllfarbe eines Symptoms verändern}
		\subsubsection{Randfarbe eines Symptoms verändern}
		\subsubsection{Titel eines Symptoms ändern}
		\subsubsection{Form eines Symptoms verändern}
		\subsubsection{Schriftart-/ größe der Beschriftung eines Symptoms ändern}
		\subsubsection{Symptome layouten}
			
		\subsubsection{Relation hinzufügen}
		\subsubsection{Relation entfernen}
		\subsubsection{Ankerpunkte einblenden}
		\subsubsection{Ankerpunkte hinzufügen}
		\subsubsection{Ankerpunkte entfernen}
		\subsubsection{Farbe einer Relation verändern}
		\subsubsection{Type einer Relation verändern}
		
		\subsubsection{Graphelemente hervorheben}
		\subsubsection{Graphelemente Hervorhebung hinzufügen}
		\subsubsection{Graphelemente Hervorhebung entfernen}
		
		\subsubsection{Graphelemente ausblenden/ einblenden}
		\subsubsection{Graphelemente Fadeout hinzufügen}
		\subsubsection{Graphelemente Fadeout entfernen}
		
		\subsubsection{Contextmenü}
	
	\subsection{Übersichtsleiste}
		\subsubsection{Filtern}
		\subsubsection{Elemente auswählen}
		\subsubsection{Contextmenü öffnen}
		
	\subsection{Zoom}
		\subsubsection{Zoom}
		\subsubsection{Zoom.Kontext}
	\subsection{Allgemeine Einstellungen}
		\subsubsection{Sprache der Benutzeroberfläche}
		\subsubsection{Sprache der Beschriftung der Graphelement}
	\subsection{Import/ Öffnen}
		\subsubsection{Neue Datei}
		\subsubsection{Datei öffnen}
		\subsubsection{GXL}
		\subsubsection{oof}
	\subsection{Export/ Speichern}
		\subsubsection{Speichern unter}
		\subsubsection{GXL}
		\subsubsection{oof}
		\subsubsection{Pdf}
		\subsubsection{Verlaufsprotokoll}
	\subsection{Drucken}
	\subsection{Analysefunktionen}
		\subsubsection{Graphmaße}
		\subsubsection{Vorgänger/ Nachfolger von Symptom(en) hervorheben}
		\subsubsection{Kürzester Pfad zw. Symptomen}
		\subsubsection{Alle Pfade zw. Symptomen}
		\subsubsection{Pfeilketten}
		\subsubsection{Konvergente Verzweigungen}
		\subsubsection{Divergente Verzweigungen}
		\subsubsection{Zyklen}
		\subsubsection{Alle Relationen eines Typs hervorheben}		
	\subsection{Verlauf}
		\subsubsection{Anzeigen}
		\subsubsection{Filtern}
	\subsection{Vorlage}
		\subsubsection{Element spezifische Vorlagereglen}
		\subsubsection{Graph spezifische Vorlagereglen}
		\subsubsection{Vorlage erstellen}
		\subsubsection{Vorlage verwenden}
		\subsubsection{Vorlage exportieren}
	\subsection{Dialogfenster}
	\newpage
	\section{Fehlermeldungen} \label{sec:fehlermeldungen}
	
	%die verschiedenen Fehlermeldungen, die der Nutzer bekommen kann erklären als subsec
	
%%%%%%%%%%%%%%%%%%%%%%%%%%%%%%%%%%%%%%%%%%%%%%%%%%%%%%%%%%%%%%%%%%%%%%
\section{Warnhinweise} \label{sec:warnhinweise}
	
	
	
	
%%%%%%%%%%%%%%%%%%%%%%%%%%%%%%%%%%%%%%%%%%%%%%%%%%%%%%%%%%%%%%%%%%%%%%
\section{FAQ}
\newpage
%%%%%%%%%%%%%%%%%%%%%%%%%%%%%%%%%%%%%%%%%%%%%%%%%%%%%%%%%%%%%%%%%%%%%%
\section{Anhang} \label{sec:anhang}	
	\subsection{Glossar}
	
	\begin{description}
		\item[Syndromansatz] Graph bla bla
	\end{description}
	
	\subsection{Abbildungsverzeichnis}
	\listoffigures
	

\newpage



%%%%%%%%%%%%%%%%%%%%%%%%%%%%%%%%%%%%%%%%%%%%%%%%%%%%%%%%%%%%%%%%%%%%%%

\end{document}


